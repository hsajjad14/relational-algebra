
\documentclass{article}

% Packages
\usepackage{amsmath, amssymb}   % math formatting & symbols
\usepackage{graphicx}           % insert graphics
\usepackage[table,xcdraw]{xcolor}
\usepackage[normalem]{ulem}
\useunder{\uline}{\ul}{}
%\usepackage{eulervm, bookman}   % fonts for math & symbols
\usepackage{mathtools}
\usepackage{fullpage}          % fullpage margins

\begin{document}
% Title 
\title{CSC343 Prep9}
\author{Haider Sajjad}
\maketitle






\begin{enumerate}
    \item % 1 
    a) \{B, D, E\} \\ \\
    b) \{A, B, C, D, E, F\} \\ \\
	c) \{A, D, E, F\} \\ \\
	d) \{B, C, D, E\} \\ \\
    e) \{A, B, C, D, E\} \\ \\
      
    \item % 2 
    a) no, closure of B, $B^{+}$ is \{B, D, E\}, A is not in the set. \\ \\
    b) yes, closure of CF, $CF^{+}$ is \{A, B, C, D, E, F\}, E is in the set. \\ \\
	c) no, closure of DF, $DF^{+}$ is \{A, D, E, F\}, B is not in the set. \\ \\
	d) no, closure of BD, $BD^{+}$ is \{B, D, E\}, C is not in the set. \\ \\
    e) yes, closure of BFC, $BFC^{+}$ is \{A, B, C, D, E, F\} , A is in the set. \\ \\
       
    \item % 3
       \textbf{A}BD \\
        $A^{+}$ = \{A,D,C,E\},\quad \quad FDs = A $\rightarrow$ D 
       	\\ \\ 
       	A\textbf{B}D \\
        $B^{+}$ = \{A,B,C,D,E\},\quad \quad FDs = B $\rightarrow$ A, B $\rightarrow$ D
       	\\ \\ 
       	AB\textbf{D} \\
        $D^{+}$ = \{A,C,D,E\},\quad \quad FDs = D $\rightarrow$ A 
       	\\ \\ 
       	A\textbf{BD} \\
        $BD^{+}$ = \{A,B,C,D,E\},\quad \quad FDs = NONE
       	\\ \\ 
       	\textbf{A}B\textbf{D} \\
        $AD^{+}$ = \{A, C, D, E\},\quad \quad FDs = NONE
       	\\ \\ 
       	\textbf{AB}D \\
        $AB^{+}$ = \{A,B,C,D,E\},\quad \quad FDs = NONE
       	\\ \\ 
       	\textbf{ABD} \\
        $ABD^{+}$ = \{A,B,C,D,E\},\quad \quad FDs = NONE
       	\\ \\ 
       	
       	Projection of S on ABD is: A $\rightarrow$ D , B $\rightarrow$ A, B $\rightarrow$ D, D $\rightarrow$ A \\ \\
       	
       	
    \item % 4\\
    		
    		An instance of R is below. The '2' that is bold and highlighted is the redundant data, because we already had the in the first row F(5) $\rightarrow$ D(2)
    		
    		\begin{table}[h!]
				\begin{tabular}{llllll}
 					\textbf{A} & \textbf{B} & \textbf{C} & \textbf{D} & \textbf{E} & \textbf{F}\\
						3 & 1 & 1 & 2                                                 & 4 & 5 \\
						4 & 7 & 2 & 3                                                 & 5 & 6 \\
						5 & 6 & 4 & {\ul \textit{\textbf{2}}} & 8 & 5
				\end{tabular}
				\end{table}
    
    
        
\end{enumerate}

		


\end{document}
