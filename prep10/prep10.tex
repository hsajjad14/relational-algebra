
\documentclass{article}

% Packages
\usepackage{amsmath, amssymb}   % math formatting & symbols
\usepackage{graphicx}           % insert graphics
\usepackage[table,xcdraw]{xcolor}
\usepackage[normalem]{ulem}
\usepackage[normalem]{ulem}
\useunder{\uline}{\ul}{}
%\usepackage{eulervm, bookman}   % fonts for math & symbols
\usepackage{mathtools}
\usepackage{fullpage}          % fullpage margins

\begin{document}
% Title 
\title{CSC343 Prep10}
\author{Haider Sajjad}
\maketitle






\begin{enumerate}
    \item % 1a
    a)
    B $\rightarrow$ EF, F $\rightarrow$ D \\\\
    b) 
    \begin{table}[htp]
\begin{tabular}{llllll}
A & B & C & D & E & F\\
3 & 1 & 1 & \textbf{2}       & 1 & \textbf{2}       \\
4 & 2 & 3 & {\ul \textbf{2}} & 3 & {\ul \textbf{2}}
\end{tabular}
\end{table}
\\
The redundancy occurs on the second line where 
we have F(2) $\rightarrow$ D(2) occuring twice.
\\\\
c) R1(BEF), R2(ABCD)
\\\\
d) R2's FDs:
\\
$A^{+}$ = ABCDEF :
A $\rightarrow$ BCD \\
$B^{+}$ = BEFD : 
B $\rightarrow$ D \\
$CD^{+}$ = ABCDEF : 
CD $\rightarrow$ ABD \\\\
R1's FDs:
B $\rightarrow$ EF\\
\\
e) Both are in BCNF, there does not exist a FD in either relation that violates that relation.
\\
\item %2
a) CB is the only key.\\\\
b) I know that nothing else is a key because the closure of any combination of attributes without C or B cannot contain every attribute. So it won't be a key.\\\\
c) Final schema: R1(ADE), R2(ABC)\\
\\
Start by creating a relation for the union of every FD: R1(ADE), R2(AC), R3(AE) \\
Since there must exist a relation which includes the superkey, create R4(ABC).\\
So relation is: R1(ADE), R2(AC), R3(AE), R4(ABC)\\ \\
Now, attributes of R2 are in R4 and attributes of R3 are in R1, so we don't need them. Leaving us with: \\
R1(ADE), R2(ABC)
    
    
    
        
\end{enumerate}

		


\end{document}
