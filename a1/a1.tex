
\documentclass{article}

% Packages
\usepackage{amsmath, amssymb}   % math formatting & symbols
\usepackage{graphicx}           % insert graphics
%\usepackage{eulervm, bookman}   % fonts for math & symbols
\usepackage{fullpage}          % fullpage margins

\begin{document}
% Title 
\title{CSC343 A1}
\author{Haider Sajjad, Muskan Patpatia}
\maketitle

\begin{LARGE}
\textbf{Relations}
\end{LARGE}

\begin{itemize}
\item Restaurant(name, owner, capacity, country)
\item Patron(PID, fname, lname, birthday)
\item Dish(DID, name, dietary)
\item Reservation(RID, PID, rname, date)
\item Order(RID, DID, number)
\item Rating(PID, rname, rating)
\end{itemize}


\begin{LARGE}
\textbf{Part 1}
\end{LARGE}



\begin{enumerate}
    \item % 1 
    Question: Report the name of the Patron that has given the highest rating to a restaurant. If there are ties,
report all of them.\\
    
        Let $r1$ and $r2$ be two Rating relations
        
        \begin{align}
        r1:= \rho \text{  r1 Rating}  \\
        r2:= \rho \text{  r2 Rating}
		\end{align}     
		
		 Let nottop be the relation of all PID's that are not the maximum
		\begin{align}
        nottop(PID):= \Pi_{(r1.PID)}\;(\sigma_{r2.rating > r1.rating \: \wedge \: r2.PID  \neq r1.PID}\;(r1 \times r2))
		\end{align} 
		Let temp be the relation of all maximums, pid's that are in Rating but not in nottop. Natural Join temp with patron to get the names of the Patrons with pids in temp.
		\begin{align}
        temp(PID):= 	(\Pi_{PID} Rating) - nottop\\
        \Pi_{fname,lname}\;(temp \bowtie Patron)
		\end{align} 
		
       
    \item % 2 
        Question: Report the name of the restaurant for which the highest number of reservations were made. If there
are ties report all of them.\\

        Can't be done with relational algebra operations. 
    \item % 3
        Question: Report the PID(s) of the Patrons(s) who reserved a spot at a restaurant, but did not order anything.\\
        
        Make an pidsOrdered relation that holds the pid of every patron who has made an order by natural joining Order with Reservation. Then subtract pidsOrdered from the set of all Reservation pids, this will give all the patron pids who have a reservation but have not ordered anything.
        \begin{align}
        pidsOrdered(RID):= \Pi_{RID}\;(Order \bowtie Reservation)\\
        (\Pi_{PID} Reservation) - pidsOrdered
		\end{align} 
       
    \item % 4 
         Question: Report the name(s) of the Patrons(s) who have made a reservation to the restaurant
named ‘Boston Pizza’ and ordered 3 of a dish called ’Margherita Pizza’.\\

Make a relation called pBP that holds the pid and rid of every patron who has made a reservation at 'Boston Pizza'. 
And a relation called mp that holds the did of a dish called ’Margherita Pizza’.

\begin{align}
       pBP(PID,RID) := \Pi_{PID, RID}\;(\sigma_{rname = \text{'Boston Pizza'}}(Reservation))\\
        mp(DID) := \Pi_{DID}\;(\sigma_{name = \text{’Margherita Pizza’}}(Dish))
		\end{align} 
		
		Let temp be the relation holding every order from Boston pizza that has 3 of any dish. Then, natural join temp and mp to get all the orders where ’Margherita Pizza’ was ordered 3 times. And natural join again with Patron to  get the names of the patrons. 
		
\begin{align}
       temp := \sigma_{number = 3}(pBP \bowtie Order))\\
       \Pi_{fname, lname}[(\Pi_{PID} (mp \bowtie temp)) \bowtie Patron]
		\end{align} 

\item % 5 
Question: Report the owner of the restaurant with the highest average rating. If there are ties, report all of them.\\

We can't find average in relational algebra.

Let r1 and r2 be two instances of Rating relation.
         \begin{align}
        r1:= \rho \text{  r1 Rating}  \\
        r2:= \rho \text{  r2 Rating}
		\end{align}     
		...

\item % 6
Question: Report the capacities of the restaurants from which patrons have so far only ordered foods with a
‘gluten-free’ dietary restriction.\\	

Make a relation called glutenfree that holds the did, of every dish that is gluten free. And make a relation called gfOrders that holds the rid of every patron who has ordered a gluten free patron.
		
		\begin{align}
        glutenfree(DID):= \Pi_{RID} (\sigma_{directory = \text{‘gluten-free’}}(Dish))\\
        temp(RID) := \Pi_{RID} (glutenfree \bowtie Order)
		\end{align} 
	Make another relation called restnames that holds the restaurant name of every restaurant that serves 'gluten-free food'. Do this by natural joining temp with Reservation. 
	Finally cross restnames and Restaurant to get the capacity of every resteraunt serving 'gluten-free'.
	
	\begin{align}
         restnames(rname):= \Pi_{rname} (temp \bowtie Reservation)\\
         \Pi_{capacity} (\sigma_{restnames.rname = Restaurant.name} (restnames \times Restaurant))
		\end{align} 
	
\item% 7
Question: Report the restaurant owner for which the very earliest reservation out of all the reservations in the
database was made. Report any ties.\\


\end{enumerate}


\end{document}
