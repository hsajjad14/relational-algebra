
\documentclass{article}

% Packages
\usepackage{amsmath, amssymb}   % math formatting & symbols
\usepackage{graphicx}           % insert graphics
%\usepackage{eulervm, bookman}   % fonts for math & symbols
\usepackage{mathtools}
\usepackage{fullpage}          % fullpage margins
\usepackage{adjustbox}
\begin{document}
% Title 
\title{CSC343 A3}
\author{Haider Sajjad, Muskan Patpatia}
\maketitle

\begin{LARGE}
\textbf{Database Design and SQL DDL}
\end{LARGE}



\begin{enumerate}
    \item % 1 
    a)
    The only key in R is CE.
       \\ \\
    b)
    Lets assume the given FD’s do not form a minimal basis. Make a set S to hold the FD's in minimal cover.
    \\ \\
    The set of FD's are: 
    $\{A\rightarrow B, CD\rightarrow A, CB\rightarrow D, CE\rightarrow D, AE\rightarrow F \}$
    \\  \\
    To find minimal basis we first start with 
    reducing left hand side of every FD X $\rightarrow$ Y where $|X|\geq 2$ in the set. Let's make a set L to hold these FD's:
    \\
    \\
    CD $\rightarrow$ A\\
    $C^{+}=C$, and $D^{+}=D$\\
    So L = \{CD $\rightarrow$ A\}
    \\
    \\
    CB $\rightarrow$ D\\
    $C^{+}=C$, and $B^{+}=B$\\
    So L = \{CD $\rightarrow$ A, CB $\rightarrow$ D\}\\\\
    CE $\rightarrow$ D\\
    $C^{+}=C$, and $E^{+}=E$\\
    So L = \{CD $\rightarrow$ A, CB $\rightarrow$ D, CE $\rightarrow$ D\}\\\\
    AE $\rightarrow$ F\\
    $A^{+}=AB$, and $E^{+}=E$\\
    So L = \{CD $\rightarrow$ A, CB $\rightarrow$ D, CE $\rightarrow$ D,  AE $\rightarrow$ F\}\\\\
    Next Let's remove any redundant FD's, first lets look at the FD's in L, if they are not redundant store them in S, as they are needed:\\\\
    CD $\rightarrow$ A, pretend this doesn't exist\\
    Then, $CD^{+}=CD$,  so we need it\\
    So S = \{CD $\rightarrow$ A\}\\\\
    CB $\rightarrow$ D, pretend this doesn't exist\\
    Then, $CB^{+}=CB$, so we need it\\
    So S = \{CD $\rightarrow$ A, CB $\rightarrow$ D\}\\\\
    CE $\rightarrow$ D, pretend this doesn't exist\\
    Then, $CE^{+}=CE$, so we need it\\
    So S = \{CD $\rightarrow$ A, CB $\rightarrow$ D, CE $\rightarrow$ D\}\\\\
    CE $\rightarrow$ D, pretend this doesn't exist\\
    Then, $CE^{+}=CE$, so we need it\\
    So S = \{CD $\rightarrow$ A, CB $\rightarrow$ D, CE $\rightarrow$ D\}\\\\
    AE $\rightarrow$ F, pretend this doesn't exist\\
    Then, $AE^{+}=ABE$, so we need it\\
    So S = \{CD $\rightarrow$ A, CB $\rightarrow$ D, CE $\rightarrow$ D, AE $\rightarrow$ F\}\\\\
    Next look at the rest of the FD's:\\\\
    A $\rightarrow$ B, pretend this doesn't exist\\
    Then, $A^{+}=A$, so we need it\\
    So S = \{A $\rightarrow$ B, CD $\rightarrow$ A, CB $\rightarrow$ D, CE $\rightarrow$ D, AE $\rightarrow$ F\}\\\\
  	Now S is the set of FD's in minimal basis, but S = the set of given FD's.\\
  	Contradiction, the given FD’s form a minimal basis.
    \\
    \\
    c) \\ 
    Minimal basis of the FD's are: $\{A\rightarrow B, CD\rightarrow A, CB\rightarrow D, CE\rightarrow D, AE\rightarrow F \}$
    \\ \\
    Using these FD's we get relations:
    $R_1(\textbf{\underline{A}}B), R_2(\textbf{\underline{CD}}A), R_3(\textbf{\underline{CB}}D), R_4(\textbf{\underline{CE}}D), R_5(\textbf{\underline{AE}}F)  $
    \\ Where the underlined letters are the keys. 
    \\ The key for this schema is $R_4$ because it has the key for R.  
    \\ \\
    d) \\
    No, all the relations in part c) are in BCNF because every FD in a relation has its left hand side be the superkey for that relation.
    \\
    \\
    \item %2 
   	a)\\\\
    This is true, here is a direct proof.
    \\ \\ 
    Assume R is a relation decomposed into relations $R_1$ and $R_2$. And assume $R_1 \cap R_2 = A$, so they share a single attribute A, and lets assume A is a key for either $R_1$ or $R_2$.\\
    \\
    We want to prove that $R_1 \bowtie R_2 = R$, proving the decomposition is lossless. 
    \\
    \\ 
    First, notice that $R \subseteq R_1 \bowtie R_2$. This is because $R_1$ and $R_2$ are decompositions of R, joining them over a shared attribute A will give a cross product of the two relations and every attribute of R will be in $R_1 \bowtie R_2$ because a cross product gives us every combination.
    \\
    \\
    Next, we will prove that $R_1 \bowtie R_2 \subseteq R$. 
    \\ \\
    First recall $R_1 \cap R_2 = A$, where A a key in either $R_1$ or $R_2$.
    This means that when natural joining $R_1$ and $R_2$, the relation with A as its key, it's tuples will appear once for every appearance of A in the join. So $R_1 \bowtie R_2 \subseteq R$ because joining over a key means no extra tuples get formed, and $|R_1 \bowtie R_2| = |R|$. 
    \\
    \\ 
    Since $R \subseteq R_1 \bowtie R_2$ and $R_1 \bowtie R_2 \subseteq R$ then $R_1 \bowtie R_2 = R$. So the decomposition is lossless.
    \\
    \\
    b)\\
    Yes, if a relation is in BCNF it is guaranteed to be in 3NF. This is because since R is in BCNF the left hand side of every FD in R will be a superkey, satisfying the 3NF requirement.
    \\ \\
    However a relation being in 3NF is not guaranteed to be in BCNF, because a relation can still be in 3NF and not satisfy the requirement that every FD's left hand side is a superkey. 
    
    \item % 3
    a) This is false. 
    \\ Take this relation for example:
  
    	\begin{table}[h]
    	\centering
		\begin{tabular}{lll}
		\textbf{A}&\textbf{B}&\textbf{C}\\
		a  & b & c \\
		a1 & b & c
		\end{tabular}
		\end{table}
		The functional dependency $A \rightarrow B$ holds because, $a \rightarrow b$ and $a1 \rightarrow b$  both hold.
		\\
		But, $b \rightarrow c$ and $b \rightarrow c1$ means the  functional dependency $B \rightarrow C$ does not hold.
		\\ \\
		So,  $A \rightarrow B$ does not imply $B \rightarrow C$.
		\\ \\
		b) This is false.
		\\ Take this relation for example:
		\begin{table}[h]
		  \centering
		\begin{tabular}{lll}
			\textbf{A}&\textbf{B}&\textbf{C}\\
			a  & b  & c  \\
			a  & b1 & c1 \\	
			a1 & b  & c2
		\end{tabular}
		\end{table}
		\\
		The functional dependency $AB \rightarrow C$ holds because ($a,b \rightarrow c$) and ($a,b1 \rightarrow c1$) and ($a1,b \rightarrow c2$) all hold.
		\\
		\\
		However, since ($a \rightarrow c$) and ($a1 \rightarrow c$),
		$A \rightarrow C$ does not hold. \\
		Also, since ($b \rightarrow c$) and ($b \rightarrow c2$),
		$B \rightarrow C$ does not hold.
		\\ \\
		So,  $AB \rightarrow C$ does not imply $A \rightarrow C$ and $B \rightarrow C$ .
		
    \item %4
    In this question, bname = "beverage name" and bsize = "beverage size". Also denote tid to be "transaction id" and cid is "customer id".\\\\
	a)
	R(city, phone, manager, bname, instock, calories, date ,price,\\
	 loyalty, bsize, tid, numTransactions, home\_store, cid)\\\\
	b)\\The functional dependencies are as follows:\\\\
		city $\rightarrow$ phone, manager\\
		city, bname, bsize $\rightarrow$ instock\\
		bname, bsize $\rightarrow$ calories\\
		tid $\rightarrow$ date, loyalty, cid, bname, bsize, city\\
		bname, bsize, loyalty $\rightarrow$ price\\
		loyalty, cid $\rightarrow$ numTransactions, home\_store\\
		R[home\_store] $\subseteq$ R[city]
	\\\\\\
	c)\begin{table}[h]
		  \centering
		\begin{adjustbox}{width=1.1\textwidth}
		\begin{tabular}{llllllllllllll}
			\textbf{city}&\textbf{phone}&\textbf{manager}&\textbf{bname}&\textbf{instock}&\textbf{calories}&\textbf{date}&\textbf{price}&\textbf{loyalty}&\textbf{bsize}&\textbf{tid}&\textbf{numTransactions}&\textbf{home\_store}&\textbf{cid}\\
			Toronto & 905 & Tom & apple & 2 & 200 & 2019-05-05 & 4.0 & 111 & r  & 1005 &4 & Toronto & 10044\\
			Toronto & 905 & Tom & apple & 1 & 200 & 2019-05-06 & 4.0 & 112 & r  & 0109 &14 & Toronto & 10045 \\	
			Toronto & 905 & Tom & green & 5 & 400 & 2019-05-06 & 5.00 & null & r & 1011 & null & null & 10011 \\
			Toronto & 905 & Tom & brown & 10 & 400 & null       & 1.0  & null & r & null & null & null & null \\
			Toronto & 905 & Tom & apple & 3  & 450 & 2019-05-08 & 5.00 & 111  & l & 1005 & 5    & Toronto & 10044 \\
			Boston  & 411 & James & purple & 9  & 200 & 2019-05-05 & 3.00 & 121  & r & 1101 & 100  & Boston  & 10043  \\
			Chicago & 403 & Henry & apple & 8 & 400 & 2019-05-05 & 4.0 & 111 & l  & 0103 &4 & Toronto & 10044\\
			Montreal & 514 & Jerry & apple & 12 & 400 & 2018-05-05 & 4.0 & 122 & l  & 4044 &2 & Toronto & 10046\\
			Montreal & 514 & Jerry & apple & 12 & 200 & 2018-05-05 & 4.0 & 123 & r  & 4059 &6 & Montreal & 10047\\
			Boston  & 411 & James & mango & 19  & 300 & 2019-06-05 & 3.00 & 124  & r & 1108 & 100  & Boston  & 10050  \\
			Chicago & 403 & Henry & mango & 8 & 600 & 2019-08-05 & 4.0 & 136 & l  & 0100 &4 & Toronto & 10049\\
			
			
		\end{tabular}
		\end{adjustbox}
		\end{table}
		\\\\
		The redundancy anomaly here would be that there are multiple tuples with duplicate information with city Toronto. For every transaction in the same city, we will get back data we already know and don't need.\\\\
		The deletion anamoly would be that if I delete manager Henry for Chicago then the complete information about the store chicago will be deleted as well.\\\\
		The update anamoly would be that if I update manager Tom for Toronto to another manager name then I will have to update other tuples for Toronto as well.
		\\\\
	d) First, to find the key of R, we will find the closure of every subset of attributes in R:\\\\
	$city^{+}$ = city, phone, manager\\
	  $city,bname,bsize^{+}$ = city, bname, bsize, phone, manager, instock\\
	  $bname,bsize^{+}$=calories\\
	  $tid^{+}$=date, loyalty, cid, bname, bsize, city, phone, manager, price, calories, instock, numTransactions, home\_store\\
	  $bname,bsize,loyalty^{+}=bname,bsize,loyalty,price,calories$\\
	  $loyalty,cid^{+}=loyalty,cid,numTransactions,home_store$\\\\
	  Therefore key is tid since all the attributes of relation R are in its closure.\\\\
	  BCNF process:\\\\
	  city $\rightarrow$ phone, manager\\
	  city is not the key therefore we need to decompose into two relations.\\
	  \textbf{S1}(city,phone,manager)\\
	  \textbf{S2}(city, bname, instock, calories, date, price, loyalty, bsize, tid, numTransactions, home\_store, cid)\\
	  For S1 city is the key and therefore S1 is in bcnf.\\
	  This FD:\\
	  city, bname, bsize $\rightarrow$ instock\\
		  Is in S2, and tid is the key in S2. This FD fails to contain it in it's left hand side. So S2 is not in BCNF.\\
		  Now decompose S2 into: \\\\
	  \textbf{S3}(city, bname, bsize, instock)\\
	  \textbf{S4}(city, bname, bsize, calories, date, price, loyalty, tid, numTransactions, home\_store,cid)\\
	  For S3 (city, bname, bsize) is the key and the projected fd is city,bname,bsize $\rightarrow$ instock
	  hence S3 is in bcnf.\\
	  For S4 key is tid and the following projected fd:\\
		bname,bsize $\rightarrow$ calories\\
		violates bcnf because it doesn't contain tid. Therefore we need to decompose S4 further.\\\\
	  \textbf{S5}(bname,bsize,calories)\\
	  \textbf{S6}(city, bname, bsize, date, price, loyalty, tid, numTransactions, home\_store, cid)\\
	  In S5 the key is bname, bsize and only projected fd bname, bsize $\rightarrow$ calories satisfies bcnf. Hence
	  S5 is in bcnf.\\
	  Now for S6 we have key is tid and the following prjected fds:
		
		bname, bsize, loyalty $\rightarrow$ price\\
	  This FD violates bcnf. Hence lets decompose S6 further.\\\\
	  \textbf{S7}(bname, bsize, loyalty,price)\\
	  \textbf{S8}(city, bname, bsize, date, loyalty, tid, numTransactions, home\_store, cid)\\
	  Now for S7 key is bname, bsize, loyalty and the only projected fd bname ,bsize, loyalty $\rightarrow$ price\\ satisfies bcnf.
	  hence S7 is in bcnf.\\
	  Now for S8 key is tid and the following projected fd \\		loyalty,cid $\rightarrow$ numTransactions,home\_store\\
	 violates bcnf property. Hence S8 needs to be decomposed further.\\\\
	   \textbf{S9}(loyalty, cid, numTransactions, home\_store)\\
	   \textbf{S10}(city, bname, bsize ,date, loyalty, tid, cid)\\
	  For S9 key is loyalty, cid and the only projected fd is loyalty, cid $\rightarrow$ numTransactions ,home\_store and this fd satisfies bcnf property.
	  hence S9 is in bcnf.\\
	  For S10 key is tid and only one FD fits in it.\\
		tid $\rightarrow$ date,loyalty,cid,bname,bsize,city\\
	  Since the key is tid and the projected fd satisfies bcnf property. Therefore S10 is is in bcnf.\\\\
	  All relations after renaming to appropriate names:\\\\
		 \textbf{juice\_stock}(\underline{city, bname, bsize}, instock)\\
		 \textbf{store}(\underline{city},phone,manager)\\
		 \textbf{beverages}(\underline{bname, bsize}, calories)\\
		 \textbf{beverage\_price}(\underline{bname, bsize, loyalty}, price)\\
		 \textbf{loyalty\_card}(\underline{loyalty, cid}, numTransactions, home\_store)\\
		 \textbf{transactions}(\underline{tid}, city, bname, bsize, date, loyalty, cid)\\\\
	e)After bcnf normalization into different relations, we can easily store informarmation about a store in the store relation without first having a purchase made at the store
	  as was the case in our previous relation where we had to have an order at a particular store before we could input information about that store in our relations.
	  \\\\
	  Now if we decide to update the manager name for a store after a chnage of manager we can simply do that in the store relation without having to wory about updation elsewhere.
	  \\\\
	  Now incase of deleting an only order at a particular store, we can simply delete the order details from the transaction relation wihout having to delete the store information 
	  as well as was the case in our previous relation where we had to delete the store info completely incase of an only order at a store.
	  \\\\
	
	f) From part d) the relations in the fruits.ddl file are:\\\\
	\textbf{juice\_stock}(\underline{city, bname, bsize}, instock)\\
		 \textbf{store}(\underline{city},phone,manager)\\
		 \textbf{beverages}(\underline{bname, bsize}, calories)\\
		 \textbf{beverage\_price}(\underline{bname, bsize, loyalty}, price)\\
		 \textbf{loyalty\_card}(\underline{loyalty, cid}, numTransactions, home\_store)\\
		 \textbf{transactions}(\underline{tid}, city, bname, bsize, date, loyalty, cid)\\\\
		 In the store table, each store is in a single city so we can identify each store with a unique city, using that as our key. Also a store must have a phone number and manager, so those can't be null.\\\\
		 In the beverages table we want to identify every beverage available, a beverage is identified by it's name and size. Also every beverage has calories, so that can't be null.\\\\
		 The juice\_stock table keeps track of the stock of beverages in each store. 
		 Every store can have different stocks of juices so it must be identified by the city (store), the beverage name, and beverage size. A juice can only be in stock if it exists, for a juice to exist it must be in the beverages table, so beverage\_size and beverage\_name in juice stock reference beverage\_size and beverage\_name in beverages, a foreign key.\\\\
		 The beverage\_prices table holds the price of every beverage. The price can be discounted if a loyalty card is used. So loyalty is in the key along with beverage\_name and beverage\_size, but since it can be null, we have a trigger to replace the null with 'None'. \\\\
		 The loyalty\_card table keeps track of the loyalty cards that customers hold, the key is a customer id and loyalty card, on the card it holds number of transactions, and home store a city which references a city from store table.\\\\
		 The transactions table holds transaction id the key, city where the transaction occurred which references cities from store table. beverage\_name and beverage\_size are a foreign key referencing  beverage\_name and beverage\_size from beverages table.

\end{enumerate}


\end{document}
